% This example is meant to be compiled with lualatex or xelatex
% The theme itself also supports pdflatex
\PassOptionsToPackage{unicode}{hyperref}
\documentclass[aspectratio=1610, 9pt]{beamer}

% Load packages you need here
% \usepackage{polyglossia}
% \setmainlanguage{american}

% IACT Tikz
\usepackage{tikz}
\usepackage{iacttikz}

\usepackage{csquotes}


\usepackage{amsmath}
\usepackage{amssymb}
\usepackage{mathtools}

\usepackage{hyperref}
\usepackage{bookmark}

\usepackage{graphicx}

\usepackage{siunitx}

% load the theme after all packages

\usetheme[
  showtotalframes, % show total number of frames in the footline
  % dark, % optional dark theme, uncomment to use
]{tudo}

% Put settings here, like
\unimathsetup{
  math-style=ISO,
  bold-style=ISO,
  nabla=upright,
  partial=upright,
  mathrm=sym,
}

\title{3D Shower Reconstruction with the Cherenkov Telescope Array}
\author[S.~Fröse]{Stefan Fröse}
\institute[WG Elsässer]{WG Elsässer \\  Faculty of Physics}
\date{22.03.2023}
\titlegraphic{\includegraphics[width=0.7\textwidth]{logos/ctapipe_logo.png}}


\begin{document}

\maketitle

\begin{frame}
  \begin{center}
    \fontsize{40}{48} \selectfont\textcolor{tugreen}{Cherenkov Telescope Array}
  \end{center}
\end{frame}

\begin{frame}{Cherenkov Telescope Array}
    \begin{minipage}{0.49\textwidth}
        \begin{itemize}
            \item Imaging Air Cherenkov Telescopes (IACTs)
            \item Large-, Medium- and Small-Sized Telescopes\\$\rightarrow$Diameter \SI{23}{\meter},\SI{12}{\meter} \& \SI{4}{\meter}
            \item Northern array site: La Palma
            \item Southern array site: Chile
            \item Energy range from \SI{20}{\GeV} to \SI{300}{\TeV}
        \end{itemize}
    \end{minipage}
    \hfill
    \begin{minipage}{0.5\textwidth}
        \centering
        \includegraphics[width=\textwidth]{build/images/telescope_render.jpg}\\
        \tiny{{\textcolor{gray}{Some credits}}}
    \end{minipage}
\end{frame}

\begin{frame}
  \begin{center}
    \fontsize{40}{48} \selectfont\textcolor{tugreen}{Extensive Air Showers}
  \end{center}
\end{frame}

\begin{frame}{Extensive Air Showers}
    \centering
    \begin{tikzpicture}[transform canvas={scale=0.8, xshift=-3.5cm, yshift=-5.5cm}]
      \clip (-6,0) rectangle (13,9);
      
      \mountainrange
      \initgamma
    \end{tikzpicture}
\end{frame}

\begin{frame}[noframenumbering]{Extensive Air Showers}
    \centering
    \begin{tikzpicture}[transform canvas={scale=0.8, xshift=-3.5cm, yshift=-5.5cm}]
      \clip (-6,0) rectangle (13,9);
      
      \mountainrange
      \initgamma
      \airshower
      \easdesc
    \end{tikzpicture}
\end{frame}

\begin{frame}[noframenumbering]{Extensive Air Showers}
    \centering
    \begin{tikzpicture}[transform canvas={scale=0.8, xshift=-3.5cm, yshift=-5.5cm}]
      \clip (-6,0) rectangle (13,9);
      
      \mountainrange
      \lightpool
      \initgamma
      \airshower
      \cherenkovdesc
      \easdesc
    \end{tikzpicture}
\end{frame}

\begin{frame}[noframenumbering]{Extensive Air Showers}
    \centering
    \begin{tikzpicture}[transform canvas={scale=0.8, xshift=-3.5cm, yshift=-5.5cm}]
      \clip (-6,0) rectangle (13,9);
      
      \mountainrange
      \lightpool
      \initgamma
      \airshower
      \telescopes
      \cameraframe
      \cherenkovdesc
      \easdesc
      \cameradesc
    \end{tikzpicture}
\end{frame}

\begin{frame}
  \begin{center}
    \fontsize{40}{48} \selectfont\textcolor{tugreen}{3D Shower Model}
  \end{center}
\end{frame}

\begin{frame}{3D Shower Model}
    \centering
    \begin{tikzpicture}[transform canvas={scale=0.8, xshift=-3.5cm, yshift=-5.5cm}]
      \clip (-6,0) rectangle (13,9);
      
      \mountainrange
      \initgamma
      \airshower
      \telescopes
      \cameraframe
      \cameradesc
      \easdesc

      \draw[cherenkovfade,rotate=-55] (-2.95cm,6.635cm) ellipse (2cm and 0.5cm);
      \draw[cherenkovfade,rotate=-55] (-2.95cm,7.135cm) -- (-2.95cm,6.135cm);
      \draw[cherenkovfade,rotate=-55] (-4.95cm,6.635cm) -- (-0.95cm,6.635cm);



      \path[
        fgcolor,
        every node/.style={font=\sffamily\small},
        ->,
        >=latex,
        xshift=\globalshift
      ] (3,6) edge[bend right] node [left] {} (2,5.445);

      \node[
        fgcolor,
        right,
        xshift=\globalshift
      ] at (3,6) {\large 3D Photosphere};
    \end{tikzpicture}
\end{frame}

\begin{frame}{3D Shower Model}
    \begin{minipage}{0.49\textwidth}
        \begin{itemize}
            \item Published by H.E.S.S. collaboration \\ \tiny{{\textcolor{gray}{"Selection and 3D-Reconstruction of Gamma-Ray-induced Air Showers with a Stereoscopic System of Atmospheric Cherenkov Telescopes", M. Lemoine-Goumard et al., 2006}}} \normalsize
            \item Assume rotational invariant Gaussian
            \item Parameters: \begin{itemize}
                    \item Polar angles of the origin $(\theta_0,\phi_0)$
                    \item Coordinates of intersection with ground $(x_0,y_0)$
                    \item Position of the barycenter $B$
                    \item Longitudinal and transverse standard deviation $(\sigma_L, \sigma_T)$
                    \item Number of Cherenkov photons $N_C$
                \end{itemize}
        \end{itemize}
    \end{minipage}
    \hfill
    \begin{minipage}{0.5\textwidth}
        \centering
        \begin{tikzpicture}
          
          \initgamma

          \draw[cherenkovfade,rotate=-55] (-2.95cm,6.635cm) ellipse (2cm and 0.5cm);
          \draw[cherenkovfade,rotate=-55] (-2.95cm,7.135cm) -- (-2.95cm,6.135cm);
          \draw[cherenkovfade,rotate=-55] (-4.95cm,6.635cm) -- (-0.95cm,6.635cm);

          \node[
            fgcolor,
            right,
            xshift=\globalshift,
            rotate=35
          ] at (0.98,5.85) {\large $\sigma_T$};
          \node[
            fgcolor,
            right,
            xshift=\globalshift,
            rotate=-55
          ] at (0.5,7) {\large $\sigma_L$};
        \end{tikzpicture}
    \end{minipage}
\end{frame}

\begin{frame}
  \begin{center}
      \fontsize{40}{48} \selectfont\textcolor{tugreen}{ctapipe - Implementation}
  \end{center}
\end{frame}

\begin{frame}{ctapipe - Implementation}
    \begin{minipage}{0.49\textwidth}
        \begin{itemize}
            \item What's ctapipe? - Low-level data processing pipeline
            \item Likelihood fit based on camera images with iminuit
            \item Cleaned camera images: Data Level 1 (DL1)
            \item Seeds: Hillas stereo reconstruction \\ $\rightarrow$ Data Level 2 (DL2)
        \end{itemize}
    \end{minipage}
    \hfill
    \begin{minipage}{0.5\textwidth}
        \centering
        \begin{tikzpicture}
            \cameraframe
        \end{tikzpicture}\\
        \tiny{{\textcolor{gray}{Anno Knierim}}}
    \end{minipage}
\end{frame}

\begin{frame}{ctapipe - Implementation}
    \centering
    \includegraphics[height=0.94\textheight]{"build/plots/reco_7.pdf"}\\
\end{frame}

\begin{frame}
  \begin{center}
    \fontsize{40}{48} \selectfont\textcolor{tugreen}{Outlook}
  \end{center}
\end{frame}

\begin{frame}{Outlook}
    \begin{minipage}{0.49\textwidth}
        \begin{itemize}
            \item New parameters $N_C$ and $(\sigma_L,\sigma_T)$ for energy reconstruction and gamma-hadron separation
            \item Compare angular resolution to Hillas stereo reconstruction
            \item Make it faster! $\mathcal{O}(\SI{1}{\second})$ per event
        \end{itemize}
    \end{minipage}
    \hfill
    \begin{minipage}{0.5\textwidth}
        \centering
        \includegraphics[trim={25.35cm 0 0 0}, clip, height=\textheight]{"build/plots/reco_7.pdf"}\\
    \end{minipage}
\end{frame}

\begin{frame}
  \begin{center}
    \fontsize{40}{48} \selectfont\textcolor{tugreen}{Thank You!}
  \end{center}
\end{frame}

\begin{frame}[noframenumbering]{Appendix}
    \centering
    \includegraphics[height=0.94\textheight]{"build/plots/reco_0.pdf"}\\
\end{frame}

\begin{frame}[noframenumbering]{Appendix}
    \centering
    \includegraphics[height=0.94\textheight]{"build/plots/reco_1.pdf"}\\
\end{frame}

\begin{frame}[noframenumbering]{Appendix}
    \centering
    \includegraphics[width=\textwidth]{"build/plots/reco_3.pdf"}\\
\end{frame}

\begin{frame}[noframenumbering]{Appendix}
    \centering
    \includegraphics[height=0.94\textheight]{"build/plots/reco_6.pdf"}\\
\end{frame}

\begin{frame}[noframenumbering]{Appendix}
    \centering
    \includegraphics[width=\textwidth]{"build/plots/reco_8.pdf"}\\
\end{frame}

\begin{frame}[noframenumbering]{Appendix}
    \centering
    \includegraphics[height=0.94\textheight]{"build/plots/reco_9.pdf"}\\
\end{frame}

\end{document}
